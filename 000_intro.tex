% !TeX spellcheck = ru_RU
% !TEX root = vkr.tex

\section*{Введение}
\thispagestyle{withCompileDate}

Архитектура набора команд (ISA) RISC-V \cite{waterman2014risc} на сегодняшний день является одной из наиболее активно развивающихся открытых архитектур.
Её открытость, модульность и поддержка со стороны научного и индустриального сообществ способствуют её распространению в широком спектре систем — от встраиваемых платформ до высокопроизводительных процессоров.

Важную роль в экосистеме RISC-V играет проект LLVM \cite{lattner2004llvm}, в котором спецификация RISC-V ISA описана с помощью декларативного языка TableGen \cite{tblgen}.
Параллельно, существует одобренный RISC-V Foundation эмулятор Sail \cite{gray2017sail}, который может эмулировать различные архитектуры, в том числе RISC-V.
Данный эмулятор используется для верификации, моделирования и формального анализа, обеспечивая строгую интерпретацию поведения инструкций на уровне архитектуры.
Таким образом, одновременно существуют две ключевые спецификации RISC-V ISA, каждая из которых описывает одни и те же инструкции, но с разных позиций.

Несмотря на наличие двух независимых спецификаций RISC-V ISA, в настоящее время не существует инструментов, позволяющих систематически сопоставлять эти спецификации и проверять их согласованность.
Это создаёт риск накопления расхождений между декларативным описанием в LLVM и формальной моделью в Sail по мере их эволюции.
Такие расхождения могут затрагивать синтаксис инструкций, ограничения на операнды, семантические свойства или поведение в особых ситуациях.
Наличие подобных расхождений может приводить к некорректной генерации кода компилятором, расхождению поведения аппаратных реализаций и снижению общей надёжности экосистемы RISC-V.

Настоящая работа посвящена валидации декларативной спецификации RISC-V ISA в LLVM относительно её формальной модели в Sail.
В работе предлагается разработать анализатор, который будет автоматически искать несогласованности спецификаций на Sail и LLVM.

% Формат из 4х частей рекомендуется в курсе Д.~Кознова~\cite{koznov} по написанию текстов.

% \begin{enumerate}
%     \item Известная информация (background/обзор).
%     \item Неизвестная информация (пробел в знаниях, \enquote{Gap}).
%     \item Гипотезы, вопросы, цели~--- \enquote{что болит}, что будет решать Ваша работа.
%     \item Подход, план решения задачи, предлагаемое решение.
% \end{enumerate}

% Последний абзац должен читаться и быть понятен в отрыве от других трёх.
% Никакие абзацы нумеровать нельзя.

% Части (абзацы) должны занять максимум две страницы, идеально уложиться в одну.

% С.-П. Джонс~\cite{SPJGreatPaper} предлагает несколько другой формат написания введения.
% Вполне возможно, что если Ваша работа про языки программирования, то его формат будет удачнее.

% Введение и заключение читают чаще всего, поэтому они должны быть \enquote{вылизаны} до блеска.

% \blfootnote{
%     Иногда рецензенту полезно знать какого числа компилировался текст, чтобы оценить актуальность версии текста.
%     В этом случае полезно вставлять в текст дату сборки.
%     Для совсем официальных релизов документа это не вполне канон.\\
%     Также здесь имеет смысл указать, если работа сделана на деньги, например, Российского Фонда Фундаментальных Исследований (РФФИ) по гранту номер такой-то, и т.п.}
