% !TeX spellcheck = ru_RU
% !TEX root = vkr.tex

\section{Обзор}
\label{sec:relatedworks}

\subsection{RISC-V}
RISC-V --- свободная и открытая архитектура процессоров, впервые представленная в 2010 году в университете Беркли, США \cite{waterman2016design}.
Создана по концепции RISC (Reduced Instruction Set Computer) \cite{jamil1995risc}, то есть обладает сравнительно небольшим набором инструкций с возможностью дополнения с помощью расширений.

\subsection{Sail}
Эмулятор архитектуры --- это инструмент, который позволяет исполнять программу для одной архитектуры на процессоре другой (или той же) архитектуры.
Одним из таких инструментов является Sail.
На языке Sail реализованы модели множества архитектур, включая X86, Arm, RISC-V и многие другие.
По моделям на данном языке Sail умеет генерировать эмулятор на \OCaml{} или \C{}.
Кроме генерации эмуляторов, Sail также позволяет верифицировать модели ISA, генерировать и исполнять автоматические тесты, а также исполнять тесты безопасности кода.
Сам компилятор эмулятора Sail написан на \OCaml{}.
Код, который отвечает за интерпретацию инструкции содержится в \texttt{execute} функциях.
Кроме этого, для каждой инструкции существует функция \texttt{encdec}, которая кодирует и декордирует инструкцию, функция \texttt{assembly}, которая отвечает за представление инструкции в виде строки.

\begin{lstlisting}[caption={Пример определения семантики исполнения инструкций addi, slti, sltiu, andi, ori, xori на Sail}, language={}, frame=single, label=itype]
mapping clause encdec = ITYPE(imm, rs1, rd, op)
    <-> imm @ rs1 @ encdec_iop(op) @ rd @ 0b0010011

function clause execute (ITYPE (imm, rs1, rd, op)) = {
  let rs1_val = X(rs1);               // read the rs1
  let immext : xlenbits = EXTS(imm);  // sign-extend imm
  let result : xlenbits = match op {  // compute the result
    RISCV_ADDI  => rs1_val + immext,  // for ADDI, do an addition
    RISCV_SLTI  => EXTZ(rs1_val <_s immext), // ...etc
    RISCV_SLTIU => EXTZ(rs1_val <_u immext),
    RISCV_ANDI  => rs1_val & immext,
    RISCV_ORI   => rs1_val | immext,
    RISCV_XORI  => rs1_val ^ immext
  };
  X(rd) = result;                     // write result to rd
  RETIRE_SUCCESS                      // successful termination
}

mapping itype_mnemonic : iop <-> string = {
  RISCV_ADDI  <-> "addi",
  RISCV_SLTI  <-> "slti",
  RISCV_SLTIU <-> "sltiu",
  RISCV_XORI  <-> "xori",
  RISCV_ORI   <-> "ori",
  RISCV_ANDI  <-> "andi"
}

mapping clause assembly = ITYPE(imm, rs1, rd, op)
  <-> itype_mnemonic(op) ^ spc() ^ reg_name(rd) ^ sep()
      ^ reg_name(rs1) ^ sep() ^ hex_bits_signed_12(imm)

\end{lstlisting}

\subsection{LLVM TableGen}

LLVM --- это инфраструктура для разработки компиляторов, созданная в 2003 году в Иллинойсском университете.
LLVM поддерживает широкий спектр целевых платформ, включая популярные x86 и ARM, так и молодые архитектуры, такие как RISC-V.
Важную роль в бэкендах LLVM играет TableGen.

TableGen --- это декларативный язык, используемый в LLVM для многих целей.
В частности TableGen используется для описания ISA целевой архитектуры.
Информация из TableGen файлов используется в бэкендах LLVM.
Файлы TableGen имеют расширение \texttt{.td}.
Инструментами LLVM можно \texttt{.td} файлы скомпилировать в \texttt{JSON}.

\begin{lstlisting}[caption={Описание инструкции addi при помощи TableGen}, language={}, frame=single, mathescape = true]
let hasSideEffects = 0, mayLoad = 0, mayStore = 0 in
class ALU_ri<bits<3> funct3, string opcodestr>
    : RVInstI<funct3, OPC_OP_IMM, (outs GPR:$\dollar$rd),
        (ins GPR:$\dollar$rs1, simm12:$\dollar$imm12),
        opcodestr, "$\dollar$rd, $\dollar$rs1, $\dollar$imm12">,
        Sched<[WriteIALU, ReadIALU]>;

let isReMaterializable = 1, isAsCheapAsAMove = 1 in
def ADDI  : ALU_ri<0b000, "addi">;
\end{lstlisting}

В описании инструкции на TableGen может, например, постулироваться, что некоторые команды процессора имеют соответствующие входные и выходные регистры, (не)могут писать в память, являются (не)коммутативными, сложность операций и т.п.

\subsection{Прототип инструмента}

Для исследования предметной области моим научным руководителем уже была начата разработка анализатора для валидации.
Данные в анализатор из LLVM поступают в виде \texttt{JSON}, код на Sail анализируется как есть.
Сам Sail реализован на языке \OCaml, и анализатор использует компилятор Sail как библиотеку.
Анализатор умеет находить исполняемые спецификации для инструкций RISC-V и запускать на них тривиальные анализы (а именно наивную проверку выходных аргументов инструкций).

% \emph{Обзор существующих решений должен быть.}
% Здесь нужно писать, что индустрия и наука уже сделали по вашей теме.
% Он нужен, чтобы Вы случайно не изобрели какой-нибудь велосипед.

% По-английски называется related works или previous works.

% Если Ваша работа является развитием предыдущей и плохо понима\-ема без неё, то обзор должен идти в начале.
% Если же Вы решаете некоторую задачу новым интересным способом, то если поставить обзор в начале, то читатель может устать, пока доберется до вашего решения.
% В этом случае уместней поставить обзор после описания Вашего подхода к проблеме%
% \footnote{Такой подход рекомендуется в работе~\cite{SPJGreatPaper}.
%     Вполне возможно, что Ваш реальный научный руководитель будет не согласен, и потребует, чтобы обзор был в начале.}.

% В обзоре вам нужно рассказать про \emph{преимущества и недостатки} того, что было сделано до Вас.
% Неправильным будет перечислять только недостатки, так как если Ваша работа хоть где-то хуже предыдущей, то рецензент будет радостно потирать руки и заваливать Вашу работу.
% Гораздо лучше, если Вы честно признаетесь в этом сами.

% \subsection{Обзор используемых технологий}

% Для технических работ обзор может обозревать продукт, в рамках которого Вы выполняете задачу, другие продукты, где решалась схожая задача, а также используемые технологии с обоснованием выбора тех, которые Вы дальше используете.
% Это всё, скорее всего, будет отдельными подразделами обзора.
% \enquote{Выбор} подразумевает наличие вариантов, поэтому опишите, из чего выбирали и почему выбрали то, что выбрали.
% Очень желательны чёткие критерии сравнения и сводная таблица в конце, где стоят плюсы и минусы рядом с каждым рассматриваемым вариантом.

% В обзоре необходимо ссылаться на работы других людей.
% В данном шаблоне задумано, что литература будет указываться в файле \verb=vkr.bib=.
% В нём указываются пункты литературы в формате \BibTeX{}, а затем на них можно ссылаться с помощью \verb=\cite{...}=.
% Та литература, на которую Вы сошлетесь, попадет в список литературы в конце документа.
% Если не сошлетесь~---  не попадёт.
% Спецификацию в формате \BibTeX{} почти никогда (для второго курса~--- никогда), не нужно придумывать руками.
% Правильно: находить в интернете описание цитируемой статьи%
% \footnote{Например, \url{https://dl.acm.org/doi/10.1145/3408995} (дата обращения: \DTMdate{2022-12-17}).},
% копировать цитату с помощью кнопки \foreignquote{english}{Export Citation} и вставлять в \BibTeX{} файл.
% Так же умеет генерировать \BibTeX{}-описания и Google Scholar%
% \footnote{Поисковая система для научных текстов Google Scholar, \url{https://scholar.google.com} (дата обращения: \DTMdate{2024-01-13})}.
% Если так не делать, то оформление литературы будет обрастать ошибками.
% Например, \BibTeX{} по особенному обрабатывает точ\-ки, запятые и \verb=and= в списке авторов, что позволяет ему самому понимать, сколько авторов у статьи, и что там фамилия, что~--- имя, а что~--- отчество.
% Google Scholar пытается генерировать описания автоматически, так что, возможно, потребуется ручная правка~--- обязательно проверьте свой список литературы.

% В обзоре и в остальном тексте вы наверняка будете использовать названия продуктов или языков программирования (например, \csharp{}).
% Для них рекоменду\-ется (в файле \verb=preamble2.tex=) за\-дать специальные команды, чтобы писать сложные названия правильно и одинаково по всему доку\-менту.
% Написать с ошибкой  название любимого языка программирова\-ния науч\-ного руко\-водителя~--- идеальный вариант его разозлить.

% \subsection{Выводы}

% Опишите явно, что читатель должен был вынести из обзора в отдельном подразделе.
